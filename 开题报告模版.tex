%-*- coding: UTF-8 -*-
% author:wanghuai
% date:2021.1.20
% Jiangxi University of Science and Technology

% ========================
%|                        |
%|         导包	          |
%|                        |
% ========================

\documentclass[hyperref,UTF8]{ctexart}
% 导入PDF的包
\usepackage{float}
\usepackage{pdfpages}
\usepackage{setspace}
\usepackage{fontspec}
\setmainfont{Times New Roman}
% 将章节序号改为中文

% Texpad 闪退,则需要将以下三行注释掉,并设置 Texpad 为使用外部编译器
\usepackage{zhnumber} 
\renewcommand \thesection {\zhnum{section}}
\renewcommand \thesubsection {\arabic{section}.\arabic{subsection}}

% 设置页边距
\usepackage{geometry}
\geometry{left=2.5cm,right=2.0cm,top=2.5cm,bottom=2.0cm}
% 设置页眉的包
\usepackage{fancyhdr}
\pagestyle{fancy}
\fancyhf{}
% 设置页码
\cfoot{\thepage}

% 设置目录内容格式
\usepackage{titletoc}
\renewcommand{\contentsname}{\fontsize{14pt}\selectfont \textbf{目\quad 录}}
\renewcommand*\contentsname{\hfill 目\quad 录 \hfill} 
\titlecontents{section}[4em]
    {
        \bfseries %字体加粗
        \zihao{4} %四号字体
        \vspace{11pt} %调整上下距离
    } 
    {\contentslabel{2em}} %调整
    {\hspace*{-2em}} %调整左右距离
    {~\titlerule*[0.8pc]{$.$}~\contentspage} %调整连接点粗细
\titlecontents{subsection}[6em]
    {
        \zihao{4}
        \vspace{11pt}
    } 
    {\contentslabel{2em}}
    {\hspace*{-2em}}
    {~\titlerule*[0.6pc]{$.$}~\contentspage}
% 设置每节标题格式
\CTEXsetup[format={\raggedright \heiti \zihao{3} \bfseries}]{section} % 一级标题三号黑体加粗
\CTEXsetup[format={\raggedright \heiti  \zihao{4} \bfseries}, indent={0pc}]{subsection} % 二级标题四号黑体加粗

% 设置参考文献的包
\usepackage{gbt7714} % 使用GB/T-7714-2015格式
% 设置参考文献行间距
\usepackage{natbib}
\setlength{\bibsep}{0.5ex}

% 取消图片注释中的冒号
\usepackage{caption}
\DeclareCaptionLabelSeparator{twospace}{\ ~}
\captionsetup{labelsep=twospace}

% 使用biblatex编译参考文献
%\usepackage{biblatex}


% ========================
%|                        |
%|         正文			  |
%|                        |
% ========================


\begin{document}
% 导入封面
%pdf修改为论文封面的PDF文件相对路径名
\includepdf[pages=-]{thesis_cover.pdf}

%--生成目录

% 取消目录的页眉
\fancyhead[L]{}
\fancyhead[R]{} 
\renewcommand{\headrulewidth}{0pt} 
% 目录两字中间空两格
\tableofcontents



\setcounter{page}{1}						% 设置当前页页码编号从1开始计数
\pagenumbering{Roman}						% 设置页码字体为大写罗马字体

% 若要在目录后添加空白页,请将下面三行注释,会出现报错可以忽略,不影响pdf生成
\newpage
\thispagestyle{empty} % 当前空白页不显示页码
\mbox
\newpage


% 正文字体(四号)、行间距(多倍行间距,1.25)
\fontsize{14}{17.5}\selectfont
% 段落间距为0
\setlength\parskip{0pt} % 取消段间距
% 设置页眉左右内容
\fancyhead[L]{江西理工大学} % 设置学院
\fancyhead[R]{信息工程学院\quad 张三} % 设置姓名
\renewcommand{\headrulewidth}{0.4pt} %设置页眉横线粗细
% 设置正文页码
\setcounter{page}{1}						% 设置当前页页码编号从1开始计数
\pagenumbering{arabic}						% 设置页码字体为小写阿拉伯字体


\section{ 第一章标题}
\subsection{二级标题}

\section{ 第一章标题}
\subsection{二级标题}

\section{ 第一章标题}
\subsection{二级标题}


\clearpage
\phantomsection										% 解决目录中超链接地址问题
\addcontentsline{toc}{section}{参考文献}				% 将参考文献添加到目录

\bibliographystyle{gbt7714-numerical}
\bibliography{math}
\end{document}
